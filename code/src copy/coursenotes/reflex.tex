\section{The Model}

talk about feedback and small error corrections

\subsection{Reflexes}

All Reflex Control programs are built up from collections of simple rules, called {\it reflexes}. These rules specify actions to be taken in response to specific situations. Consider what happens when you touch a hot frying pan. Without thinking, you will react by pullying your hand away. A robot can be programmed with similar rules, such as one which detects that the robot has collided with a wall and directs the robot to turn away from it.

Reflexes can be written as if-then rules which connect a stimulus to a response. The stimulus corresponds to a particular sensor state which might arise during the robot's operation. The response usually takes the form of a direct command to the actuators, though as will be discussed later, it can also transfer control to another behavior. For example, the reflex,

\begin{quote}
{\bf if} the left side bumper is pressed {\bf then} turn right
\end{quote}

will cause the robot to turn away when it scrapes up against a wall.

\subsection{Behaviors}

{\bf wall_follow_left} while front bumper is not pressed
  if the left side bumper is pressed then turn right
  otherwise turn left

\subsection{Scripts}

\begin{enumerate}
\item wall_follow_left
\item right_turn
\item wall_follow_left
\end{enumerate}

\subsection{Reactive Algorithms}