\subsection{Potentiometers}

\begin{figure}[htbp]
 \centerline{\epsfysize = 2.0in\epsffile{sensor/pot2.eps}}
 \caption{Potentiometers}
 \label{pot}
\end{figure}

Potentiometers, often called ``pots,'' are variable resistors. They come in a variety of shapes and sizes, but can be grouped into two categories: rotary and linear.

Rotary pots have a knob which can be turned to vary the resistance. By wiring the two outside pins to power (Vcc) and ground (GND) and the center tap to signal as shown in Figure \ref{pot}, the pot can be used to measure angles. They are very well-suited to measuring angles of joints in the robot.

Linear pots are very similar to rotary pots, except that they have a slider which changes the resistance. As the slider is moved back and forth, the output value changes, allowing motion in a straight line to be measured. Linear potentiometers should be wired as shown in Figure \ref{pot}.
