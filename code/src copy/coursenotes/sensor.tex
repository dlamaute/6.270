\chapter{Sensors}
\begin{quote}
The concept of a sensor should already be familiar to you. You have an
array of sensors which you use to feel, see, hear, taste, and
smell. You rely on these senses for just about everything you
do. Without them, you would be incapable of performing even the most
simple tasks.

Robots are no different. Without sensors, they are merely machines,
incapable of adapting to any change in the environment. Sensors give
your robot the ability to collect information about the world around
it and to choose an action appropriate to the situation.

After reading this chapter, you should take some time to play with
your sensors. Wire at least one of each type and learn how it works,
what values it returns, and under what conditions it will production
those values. Every sensor has its own little quirks, and only through
experimentation will you acquire the expertise necessary to integrate
them into your robot.
\end{quote}

\section{Digital Sensors}
Digital sensors work a lot like light switches. The switch can either
be in the ``on'' position or the ``off'' position, but never in
between. Even if you hold it in the center, the light will be either
on or off. When a digital sensor is on, it returns a voltage which the
controller interprets as a value of one. When it is off, the value is
zero.

\begin{figure}[htbp]
\begin{center}
\includegraphics{sensor/digsense.eps}
\caption{Digital Sensor Circuit}
\label{digsense}
\end{center}
\end{figure}
All digital sensors can be modelled as if they were switches. When
plugged into a sensor port, digital sensors resemble one of the
circuits shown in figure \ref{digsense}. When the switch is closed,
electrical current flows freely through it, and the output is pulled
down to GND. When the switch is open, the pullup resistor causes the
signal line to float to Vcc. While Vcc usually represents a logic one,
the value is inverted in software so that the value one represents the
situation where the sensor is activated.
\subsection{Switches and buttons}

\begin{figure}[htbp]
 \centerline{\epsfysize = 2.0in\epsffile{sensor/switches.eps}}
 \caption{Switches and buttons}
 \label{switches}
\end{figure}

Switches and buttons are probably the easiest and most intuitive
sensors to use. They are digital in nature and make great object
detectors as long as you are only worried about whether the robot is
touching something. Fortunately, this is usually enough for detecting
when the robot has run into a wall or some other obstacle. They can
also be used for limiting the motion of a mechanism by providing
feedback about when to stop it.

Switches and buttons come in a wide variety of styles. Some have
levers or rollers. Some look very much like computer keys. Some are
computer keys. Whatever they look like, it should be obvious which of
your sensors are switches.

Switches have two important properties which describe how they are
wired inside: number of poles and number of positions (throw). The
number of poles tells how many connections get switched when the
switch is activated. The throw represents how many different positions
the switch can be placed into. The most common types are SPST (single
pole, single throw) and SPDT (single pole, double throw). Most buttons
fall into the SPST category.

An SPST switch has two terminals which are connected when the switch
is activated and disconnected otherwise. An SPDT switch has three
terminals: common (labelled ``C''), normally open (``NO''), and
normally closed (``NC''). When the switch is activated, common is
connected to normally open, and when it is not, common is connected to
normally closed. An SPDT switch can be used as an SPST switch by
ignoring the normally closed terminal.

Switches and buttons should be wired as shown in Figure
\ref{switches}. SPST switches are not polarized, so it does not matter
which terminal is connected to signal. SPDT switches, when not used as
SPST switches, should have the common terminal connected to signal.


\section{Resistive Analog Sensors}

\begin{figure}[htbp]
\begin{center}
\includegraphics{sensor/ressense.eps}
\caption{Resistive Analog Sensor Circuit}
\label{ressense}
\end{center}
\end{figure}

Resistive sensors change resistance with changes in the
environment. When plugged into a sensor port, the sensor and pullup
resistor form a voltage divider which determines the voltage at the
signal input as shown in figure \ref{ressense}. When the resistance of
the sensor is high, little current flows through the circuit, and the
voltage across the pullup resistor is small, causing the signal
voltage to approach Vcc. When the sensor's resitance is low, more
current flows and the pullup resistor causes the signal voltage to
drop.

\subsection{Potentiometers}

\begin{figure}[htbp]
 \centerline{\epsfysize = 2.0in\epsffile{sensor/pot2.eps}}
 \caption{Potentiometers}
 \label{pot}
\end{figure}

Potentiometers, often called ``pots,'' are variable resistors. They come in a variety of shapes and sizes, but can be grouped into two categories: rotary and linear.

Rotary pots have a knob which can be turned to vary the resistance. By wiring the two outside pins to power (Vcc) and ground (GND) and the center tap to signal as shown in Figure \ref{pot}, the pot can be used to measure angles. They are very well-suited to measuring angles of joints in the robot.

Linear pots are very similar to rotary pots, except that they have a slider which changes the resistance. As the slider is moved back and forth, the output value changes, allowing motion in a straight line to be measured. Linear potentiometers should be wired as shown in Figure \ref{pot}.


\section{Transistive Analog Sensors}

All transistors have three leads, the base, collector, and
emitter. The voltage level present at the base determines how much
current is allowed to flow from the collector to the emitter. This is
easy to visualize in terms of a water faucet. As the knob (base) is
turned, water is allowed to flow through the faucet.

Transistive sensors are analog and work just like regular transistors,
except that the base is replaced with an element sensitive to some
stimulus (usually light). When the sensor is exposed to this stimulus,
the faucet opens, and current is allowed to flow from the collector to
the emitter.

\begin{figure}[htbp]
\begin{center}
\includegraphics{sensor/trasense.eps}
\caption{Transistive Analog Sensor Circuit}
\label{trasense}
\end{center}
\end{figure}

Figure \ref{trasense} shows a circuit diagram of a phototransistor
plugged into a sensor port. When the sensor is in the dark, no current
flows through the circuit. This causes the reading on the sensor port
to be pulled up to Vcc through the resistor. As the light level
increases, however, current begins to flow from Vcc through the
resistor to GND. The current causes a voltage drop across the resistor
which decreases the voltage measured at the port. When the transistor
is fully open, the measured voltage will hover around GND.

\subsection{IR LED and Phototransistor}

\begin{figure}[htbp]
 \centerline{\epsfysize = 2.0in\epsffile{sensor/ptransistor.eps}}
 \caption{IR LED and Phototransistor}
 \label{ledptransistor}
\end{figure}

The IR phototransistor's output characteristics are perfect for use with
the analog sensor ports, and in some cases, can be relied upon to produce
digital signals. It responds very well to the accompanying LED, but
barely responds to visible light. Since the components are tuned to
work with each other, best results will be achieved when they are used
together, as shown here.

The IR LED (part \#LTE-4208) has a clear package, and its two leads have different lengths.  The shorter lead on the flat side of the LED package is the cathode, and
 should be connected with the 330\ohm resistor to ground.  

The IR phototransistor (part \#LTR-3208E) also has two leads, but its package is dark.  The flat side of the phototransistor corresponds to the collector, and should be connected to the Signal header pin.  (The longer lead on the rounded side is the emitter, and should be connected to ground, as shown above).

Because the phototransistor is so sensitive, it might be helpful to wrap
some black heat shrink or electrical tape around it if you're having difficulty getting distinct readings for light and dark.  This also useful when
looking for light in a particular direction.

Once again, realize that the LED does not have to be perpetually on
during the 90 seconds the robot competes.  The LED can be plugged into
a motor output to allow differential light measurements.  And remember
any number of LEDs can be plugged into a single motor port.

These 5mm LEDs and phototransistors are particularly convenient when working with LEGO
because they are just the right size to fit into the LEGO axle holes.
It is very easy to mount this sensor into your robot when constructing
breakbeam sensors and shaft encoders.

\subsection{Breakbeam Sensor Package}

\begin{figure}[htbp]
\begin{center}
\epsfysize = 2.0 in
\epsfbox{sensor/break.eps}
 \caption{Breakbeam sensor package}
 \label{break}
\end{center}
\end{figure}

The breakbeam sensor package is composed of an infrared LED and a phototransistor which is sensitive to the wavelength of light emitted by the LED. The two components are mounted in the package so that they face each other with a gap in between them. 

The sensor should be wired as shown in Figure \ref{break}. As usual with LEDs, a 330\ohm resistor will be needed to limit the amount of current used to light it. Be sure to orient the sensor correctly, so that you do not confuse the phototransistor half with the LED half. The markings vary from one package to the next, but usually include one or more of the following:

\begin{enumerate}
\item an ``E'' (emitter) for the LED and a ``D'' (detector) for the phototransistor marked above each component.
\item arrows on the top of the package which point towards the phototransistor side.
\item a notch on the LED side of the package.
\end{enumerate}

The sensor is valuable for detecting the presense of opaque objects. Normally, light from the LED shines on the phototransistor, but when a object blocks the path, the phototransistor only sees darkness. This can be useful in constructing mechanisms which must be stopped after moving a certain distance. Also, shaft encoders can be built by using the sensor to count the number of holes in a wheel as it rotates.

Although the breakbeam sensor is analog, it can often be used as a digital sensor. In most applications, the use of the sensor is digital in nature and involve measuring whether the light is blocked or not blocked. Conveniently, the sensor's output values for these two situations are valid digital signals, so the sensor can be used in a digital application.

\subsection{Sharp Distance Sensor}

\begin{figure}[htbp]
 \centerline{\epsfysize = 2.0in\epsffile{sensor/distance.eps}}
 \caption{Sharp Distance Sensor}
 \label{distance}
\end{figure}

The Sharp GP2D12 Distance Sensor is capable of precisely measuring
distances to walls or objects, using near-infrared light.
Commercially, devices like these are used in a variety of places, from
automatic toilets and sinks to photocopiers; your robot can use it to
follow walls and avoid obstacles. This sensor is the only one which can detect things more than a few inches away---it has a useful range from about~$6.5''$ to several feet.
The package has two parts:

\begin{enumerate}
\item an emitter that emits a narrow beam of barely visible light; and
\item an array of detectors that measure the angle of the spot the
      emitter projects on the wall.
\end{enumerate}

Unlike other devices used in 6.270, this device supplies an analog
value on its own, without the use of a pull-up resistor.  Therefore,
to use the sensor, you must make a slight, reversible modification to
the Handy Board---cut the trace on the bottom of the main board that
connects an analog input to its pull-up resistor.  Please see a member
of the staff for help with this procedure.

The sensor does not give you the distance in any type of standard
unit.  Instead, the value read on the analog input it is connected to
varies smoothly from~0 to~255 as the distance decreases.  Tests have
found that the function
$$ f[n] = -3.16'' + { 950.'' \over 8.58+n } $$
gives an accurate estimate of the distance as a function of the analog
value~$n$, and that the reading is fairly independent of lighting,
object color, or battery power level.  However, the numeric parameters
may have to be recalibrated for each individual sensor if an accurate
measurement is desired.


\section{Gyroscopes}

Analog Devices and Intempco provide surface-micromachined integrated circuit gyroscopes mounted 
on Happyboard-compatible printed-circuit boards. The gyro can be used to measure your robot's 
rate of turn or (with some numerical integration) angular orientation, which can be a signicant
aid to navigation. These devices integrate the mechanical parts of the gyro along with the
necessary circuitry on a single chip, so they are very small (about 7mm square). To make them
easier to handle and use, the gyros are supplied on small printed-circuit boards
containing all necessary external components to which you solder pins.  The pins plug directly
into the Happyboard sensor input connector.

Your gyro board may have one of a few different gyros.  They will be one of the ADXRS300, ADXRS610,
ADXRS622, or AD22307.  The most significant difference between them is the full-scale range (or
sensitivity), which will be $250\,^{\circ}\mathrm{/s}$ or $300\,^{\circ}\mathrm{/s}$, or $5-7\mathrm{mV/}\,^{\circ}\mathrm{/s}$.  The data sheet for your gyro is 
available at www.analog.com.  The AD22307 is a specially-branded version of the ADXRS622. 

A gyro measures rate of turn.  Our gyros operate on a single 5V power supply, as supplied by the
Happyboard's analog input connectors.  They produce a single voltage output in the range of 0 to 5V,
so they are wired and used like most other three-wire sensors. They are not, however, ratiometric to
the power supply like accelerometers or potentiometers. The output of the gyro when
it is not rotating is nominally 2.5V, midscale on the Happyboard's A/D converter. The
nominal sensitivity of the gyro is 5-7mV per degree per second of rotation, depending on which
gyro you have. You should calibrate your gyro's sensitivity, and make sure that the robot never
turns faster than the full-scale range.  Calibrate the gyro's offset at the beginning of each run.

Some gyro boards may say 'MIT Noisy Gyro' on them.  This doesn't mean that the gyro is noisy; rather 
that noise is explicitly injected into the signal path to improve the effective resolution of the A/D 
converter.  This was more important on the older 6.270 controllers which had an
8-bit A/D converter, but has no detrimental effect when used with the Happyboard.

To get from rate of turn to swept angle, it is necessary to take a series of readings 
of the gyro's output and do a simple numerical integration. Techniques and code for 
performing the integration will be presented in lecture and in a handout.

Figure \ref{gyro1} shows the mechanical configuration of the gyro board and the
necessary connections.  To operate properly, the gyro board should be
mounted in an approximately horizontal position (parallel to the surface of the earth), 
assuming that you intend to measure the turn angle (yaw) of your robot.  It can be
plugged directly into the Happyboard sensor input connector, or it can be fastened in place 
with double-sided-sticky tape elsewhere on your robot.

%Contacts (all at Analog Devices):

%\noindent Jack Memishian \verb^(john.memishian@analog.com)^\\
%Mark Nelson \verb^(mark.nelson@analog.com)^\\
%Howard Samuels \verb^(howard.samuels@analog.com)^\\

\epsfysize = 2.0in
\begin{figure}[htp]
\begin{center}
\epsfbox{sensor/gyro1.eps}
\caption{Mechanical configuration of the gyro board.}
\label{gyro1}
\end{center}
\end{figure}






